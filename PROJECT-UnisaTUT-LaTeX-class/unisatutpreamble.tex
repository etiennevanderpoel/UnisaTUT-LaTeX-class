%% LaTeX oreamble to include in all UNISA tut letters
%%

%% SET path to graphics
\graphicspath{ {./} }
%\graphicspath{ {./images/} }


%% LOAD packages and set options

\usepackage{
	acronym,
	amsfonts,  			% blackboard bold symbols e.g. for real numbers
	amsmath,			% for any decent maths
	amssymb,			% AMS symbols
    array,			
	comment,			% selective in/exclusion of text
	enumitem,
%	etoolbox,			% e_Tex toolbox
	framed,             % nicer frames
	gensymb,
	graphicx,  			% include jpeg or pdf pictures
%	hyperref,			% to use URLs in the text
%	ifthen,				% logical flow control
    karnaugh-map, 		% Karnaugh maps
	lineno,				% add line numbers to text
	listings,           % to show code listings
	longtable,			% multi-page tables
	mathtools,			% ditto
	microtype,			% perform various microtypesetting tricks
    pgfmath,
	subfigure,			% multiple figures in one
	textcomp,
	tikz,				% Tikz figures and diagrams
	tikz-3dplot,		% 3D plots in Tikz
%	verbatim,			% verbatim text and multiline comments
	wrapfig,			% wrap text around figures
	xcolor,             % pre-defined colour names
	xfrac,				% better-looking fractions - uses \sfrac
	}

\usepackage[
    colorlinks=true,    % use coloured hyperlinks
    breaklinks=true,    % break links across lines
    ]{hyperref}




%% ADJUST some dimensions

%% add space between paragraphs
\setlength{\parskip}{12pt plus6pt minus4pt}




%% DEFINE new commands or re-define existing commmands

%% red line across page to use as editing marker
\newcommand{\redline}{
\textcolor{red}{\noindent\makebox[\linewidth]{\rule{\paperwidth}{10pt}}}}

%% thin black line across the width of the text
\newcommand{\textwidthline}{
\noindent\makebox[\linewidth]{\rule{\textwidth}{0.4pt}}}


%% command to add an open line
\def\wl{\par \vspace{\baselineskip}}

%% 
\newcommand{\var}[2]{% 
   \newlength{#1} 
   \setlength{#1}{#2} 
} 





%% DEFINE specific colour names
\definecolor{colour_coffee}{RGB}{219,144,71}
\definecolor{colour_airforceblue}{RGB}{0.36, 0.54, 0.66}
\definecolor{shadecolor}{rgb}{1.0,0.8,0.3}





%% TIKZ stuff

%% TIKZ libraries to load
\usetikzlibrary{
    automata,
    shapes,
    shapes.geometric,
    shapes.callouts,
    patterns,
    calc,
    intersections,
    angles,
    quotes,
    arrows.meta,
    arrows,
    decorations,
      decorations.markings,
      decorations.text,
      decorations.pathmorphing,
    positioning,
    }


%% set up TIKZ values and styles
\tikzset{
%% global TIKZ settings
	x=1cm,
	y=1cm,
	axis/.style={
		very thick,
		-latex,
		>=stealth'
	},
%% arrow to use in graphs
	grapharrow/.style={
		-latex,
		>=stealth'
	},
%% red triangle with black border
	triangle/.style={
		line width=0.5pt,
		draw=black,
		fill=red!60,
		shape border rotate=#1,
		isosceles triangle,
		isosceles triangle apex angle=60,
		minimum height=0.2cm,
		minimum width=0.25cm,
		isosceles triangle stretches,
		inner sep=0pt
	},
%% thin axis
	thinaxis/.style={
		thick
	},
%% black triangle
	triangle/.style={
		line width=0.3pt,
		draw=black,
		shape border rotate=#1,
		isosceles triangle,
		isosceles triangle apex angle=60,
		minimum height=0.2cm,
		minimum width=0.25cm,
%		isosceles triangle stretches,
		inner sep=0pt
	},
%% triangle with point up
	trianglepointup/.style={
		triangle=+90
		},
%% triangle with point down
	trianglepointdown/.style={
		triangle=-90
	},
%% node representing negative values
	negnode/.style={
		trianglepointdown,
		fill=red!60
	},
%% node representing positive values
	posnode/.style={
		line width=0.5pt,
		draw=black,
		fill=green!60,
		rectangle,
		minimum height=0.2cm,
		minimum width=0.2cm,
		inner sep=0pt
	},
%% node representing hypothesis in decision tree learning
	hypothesis/.style={
		rectangle,
%		fill=blue!10,
		inner sep=1pt,
		minimum size=4mm
	},
%% empty node
	nodenoborder/.style={
		rectangle,
%		fill=blue!10,
		inner sep=1pt,
		minimum size=4mm
	},
%% decision tree node
	dectreenode/.style={
		rectangle,
		draw=black,
		line width=0.75pt,
		fill=white,
		inner sep=2pt,
		minimum height=0.5cm
	},
%% decision node
    decnode/.style={
        circle,
        draw=black,
        line width=0.75pt,
        fill=white,
        inner sep=1pt,
        minimum size=0.8cm
    },
%% leaf node in graph
	leafnode/.style={
		rectangle,
		draw=black,
		line width=0.75pt,
		fill=white,
		rounded corners=1mm,
		inner sep=1pt,
		minimum size=0.8cm
	},
%% leaf node in decision tree
	leaftreenode/.style={
		rectangle,
%		draw=black,
%		line width=0.75pt,
		fill=white,
%		rounded corners=1mm,
		inner sep=2pt,
		minimum size=0.8cm
	},
%% full line arc in decision tree
	decarrowfull/.style={
		-latex,
		>=stealth',
		line width=1.2pt
	},
%% dashed line arc in decision tree
	decarrowdashed/.style={
		-latex,
		>=stealth',
		dashed,
		line width=1.2pt,
	},
%% dotted arrow
	arrowdotted/.style={
		-latex,
		>=stealth',
		dotted,
		line width=1.2pt,
	},
%% speech bubble for comments or details in diagrams
	notice/.style={
		draw,
		fill=yellow!30,
		rectangle callout,
%		callout relative pointer={#1}
		callout absolute pointer={#1}
	},
%% vector
	vecarrow/.style={
		line width=3pt,
		decoration={markings,mark=at position 1 with 
			{\arrow[line width = 3pt,fill=white]{open triangle 60}}},
		double distance=0.0pt,
		shorten >= 5.5pt,
		preaction = {decorate},
		shorten >= 3pt,
		postaction = {
			draw,
			line width=2.5pt,
			white,
			shorten >= 15pt}
	},
%% hyperplane line for 2D neural networks 
	hyperplane/.style={
		draw=blue,
		line width=3pt,
		opacity=0.3
	},
%% node in perceptron neural network 
	perceptronnode/.style={
		circle,
		draw=black,
		line width=0.75pt,
		fill=white,
		inner sep=2pt,
		minimum size=2cm
	},
%% input node in perceptron neural network
	perceptroninput/.style={
		circle,
		draw=black,
		line width=0.75pt,
		fill=black!50,
		inner sep=0pt,
		minimum size=0.75cm
	},
%% circle node
    circnode/.style={
        line width=0.5pt,
        draw=black,
        fill=blue!60,
        circle,
        minimum height=0.2cm,
        minimum width=0.2cm,
        inner sep=0pt
    },
}



%%%%%%%%%%%%%%%%%%%%%%%%%%%%%%%%%%%%%%%%%%%%%%%%%%%